\chapter{Results \& Discussion}
\label{chapter:Results}

The evaluation of the SPLASH system demonstrates a solid alignment between the technical implementation and the real needs of its end users. Both quantitative and qualitative analyses suggest that the system fulfills its usability goals and provides a strong foundation for future iterations. Moreover, the results highlight how a user-centered design approach—combined with targeted technological features—can improve operational efficiency and safety in coastal environments.

\section{Achievement of Usability Goals}
A core objective of the SPLASH project was to deliver an intuitive and effective interface for diverse user groups. The system achieved a System Usability Scale (SUS) score of 77\%, exceeding the target threshold of 70\%, and confirming the following points:

\begin{itemize}
    \item The iterative, user-centered design process successfully incorporated feedback from key stakeholders, including lifeguards, beachgoers, and business owners.
    \item The interface proved intuitive and accessible across all user profiles, supporting a wide range of tasks from hazard reporting to real-time monitoring and team coordination.
\end{itemize}

These results indicate that the SPLASH system is not only usable, but also well-received by its intended audience, establishing a strong baseline for potential real-world deployment.

\section{Key Findings from Usability Tests}
The usability testing process uncovered several strengths of the system, validating many of the design decisions made throughout the project:

\begin{itemize}
    \item \textbf{Interactive Map Functionality}: Participants praised the clarity and usefulness of the interactive map, especially in visualizing:
    \begin{itemize}
        \item Real-time environmental data (e.g., weather, tides)
        \item Hazard zones and incident reports
        \item Lifeguard and resource positions
    \end{itemize}
    This functionality was central to decision-making and increased situational awareness for all users.

    \item \textbf{Alert and Notification Systems}: The system's notification framework—including GPS-based alerts, lost child tracking, and real-time hazard warnings—was consistently identified as critical by test participants. These features provided:
    \begin{itemize}
        \item Immediate, actionable feedback to both lifeguards and beachgoers
        \item A sense of security and control, especially for guardians and supervisors
    \end{itemize}

    \item \textbf{Data Integration and Historical Analysis}: The inclusion of historical incident data and beach metrics allowed users—particularly supervisors—to:
    \begin{itemize}
        \item Identify high-risk areas and trends
        \item Improve planning and preventive strategies
        \item Justify resource allocation based on past data
    \end{itemize}
    This proved especially valuable for lifeguard coordinators and decision-makers.
\end{itemize}

Overall, the tests reinforced the importance of combining real-time data with historical insights and intuitive user interactions. The system was recognized as a valuable tool not only for emergency response, but also for enhancing day-to-day beach management and communication.

