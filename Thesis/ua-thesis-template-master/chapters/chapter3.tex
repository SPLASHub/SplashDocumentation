\chapter{State of the Art and Market Analysis}
\label{chapter:sota}

This chapter merges the current state of technology in beach safety with a market analysis, focusing on the SPLASH project and its potential impact.

\section{Related Projects Implemented in Portugal}
Several digital initiatives have been implemented in Portugal to improve beach safety, environmental monitoring, and user experience. Below are some notable examples that serve as inspiration or complementary efforts to the SPLASH project.
\subsection{MEO Beachcam}
MEO Beachcam is a digital platform that provides real-time information about the beaches of mainland Portugal.
\\
\indent The platform provides around-the-clock video transmission of the beaches through a network of cameras. In addition to the live video feeds, MEO Beachcam provides extensive meteorological data, including: weather conditions, temperature, wind strength and direction, sea state, and UV index. In addition to being accessible on the platform, the aforementioned data is also available to security forces and rescue services for monitoring and emergency response. In addition to oceanographic buoys and weather stations for data collection, the platform also employs artificial intelligence to analyze wave patterns and better predict weather conditions. 
\\
\indent The project is committed to expand and enhance its coastal monitoring infrastructure. Future plans for development include the expansion of the existing camera network and the deployment of drone technology to capture aerial imagery, all of which while seeking to increase the number of monitored beaches throughout mainland Portugal.

\subsection{Info Praia}
Info Praia is a mobile app developed by the \ac{apa}  with the purpose of providing actualized information during the bathing season on many of the beaches of Portugal\citep{infopraia1}.
\\
\indent It provides information on beach occupation and occupation prediction based on data from registered homologue days. The data used for the daily weather forecast is provided by the Instituto Português do Mar e da Atmosfera, I.P (IPMA, I.P). The data on the tides are provided by the Instituto Hidrográfico\citep{infopraia2}. The water quality is monitored by the APA at least four times for each bathing season to verify bathing suitability and display suitability in the app. Furthermore, is displayed by the app real-time video transmitted by the MEO Beachcam platform analyzed previously.
\subsection{Praia 5G}
Praia 5G is a joint initiative between NOS and Almada city council to launch the first 5G beach in Portugal. This project has the objective of demonstrating how the 5G network technology, artificial intelligence and video analysis can transform the security and management of beaches\citep{praia5g1}.
\\
\indent The beach was equipped with a 5G network to support all the functionalities implemented. One of the functionalities is the automated monitorization of persons, objects and vessels. This beaches allowed the demonstration of the detection of risk situations, automated alerts for the responsible entities and it uses artificial vision cameras in vigilance towers that process in real time the images through machine learning algorithms allowing immediate response in emergency situations\citep{praia5g1}. Another functionality is the monitorization and management of the beach occupation which includes automatic counting the number of visitors and their respective entry and exit times and the capacity to estimate the concentration of people per area of the beach\citep{praia5g1,praia5g2}. Lastly, the management of residues is elaborated by the readings of sensors that measure the garbage bins capacity\citep{praia5g2}.
\\
\indent This project has the future perspective of implementing functionalities such as drowning detection, children and lost objects localization through artificial intelligence. Also, it is intended to use drones for real time data of weather conditions and to increase the sustainability expand the system to detect garbage in the sea with the possibility of the use of cleaning robots\citep{praia5g2}.


%\section{Others Related Projects Implemented Internationally}
\section{Technologies}
\subsection{React}
React is a JavaScript library used for building user interfaces, particularly for single-page applications. Its core design is centered on the concept of reusable components, which are isolated, self-contained pieces of UI that can be composed to build complex interfaces. A key characteristic of React is its use of a Virtual DOM. This feature improves application performance by creating an in-memory representation of the UI, calculating the minimal changes needed, and then updating the real DOM efficiently.  
The library employs JSX, a syntax extension that allows developers to write HTML-like code within JavaScript, which improves code readability and maintainability. React also enforces a unidirectional data flow, where data is passed down from parent to child components, making application state more predictable and easier to debug. For this project, the React frontend was developed using modern tooling, including the Vite build tool for a fast development server and optimized production bundles. The application's functionality was extended with specific libraries, including react-router for client-side navigation and Leaflet with react-leaflet to render the interactive maps required for location tracking.  
React offers several advantages for frontend development. Its component-based architecture inherently promotes code reuse, which can significantly speed up the development process. The use of a Virtual DOM generally results in high rendering performance, leading to a responsive user experience, especially in applications with dynamic data. Furthermore, React is supported by a massive and active community, which ensures a vast ecosystem of third-party libraries, tools, and extensive documentation.  
There are also trade-offs to consider when using React. As it is fundamentally a library for the ‘view’ layer, it does not include built-in solutions for more complex application concerns like global state management or client-side routing. Consequently, projects often require integrating and managing multiple auxiliary libraries, which can increase complexity and maintenance overhead. This dependency on a wider ecosystem can present a learning curve for developers new to the platform.  
React was selected for this project because its modular architecture and performance characteristics were well-suited to the development of a high-performance, interactive user interface. The ability to create reusable components and leverage a rich ecosystem of specialized libraries, such as those for mapping and routing, was critical for delivering the required features within the project's timeline. Although a reliance on external libraries was a known consideration, the development speed and robust community support offered by the React ecosystem were deemed to outweigh the potential complexities.

\subsection{Web Bluetooth API}
This section presents the technologies employed for this project. The Web Bluetooth API allows a web page running JavaScript to act as a \ac{BLE} Central device, facilitating direct discovery and communication with nearby \ac{BLE} peripherals without requiring a native companion application. While this approach significantly streamlines integration of the proof-of-concept wearable, the API remains in an experimental stage under active development by the Web Bluetooth Community Group and is not yet a W3C standard. For security, browsers support it only in secure contexts over HTTPS, and any device-request function must be initiated through a genuine user gesture such as a click or touch event, preventing unwanted or malicious device prompts. Current compatibility extends across Chromium-based browsers including Chrome on desktop and Android, Edge, and Opera while key browsers like Firefox and Safari remain without support. Given its draft status and the associated security and compatibility limitations, the API is best suited for prototyping or controlled deployments rather than widespread production release. Nevertheless, SPLASH leverages the browser’s native \ac{BLE} functionality to connect directly to the bracelet hardware without native code, accelerating development and validating the SPLASH ecosystem vision with minimal complexity\cite{webble_1,webble_2}.

\subsection{C\# and ASP\.NET Core}
ASP .NET Core, together with the C\# language, constitutes a lightweight, modular, and cross-platform framework for building high-performance RESTful APIs. It runs reliably on Windows, Linux, and macOS thanks to the Kestrel web server and the efficient \ac{CPU}. The middleware pipeline enables precise configuration of HTTP request handling, and built-in dependency injection supports clean architectural patterns without requiring additional libraries. Asynchronous programming with async/await ensures non-blocking I/O operations, significantly enhancing scalability under concurrent workloads. Benchmarks consistently demonstrate that .NET 6 and later versions deliver throughput on par with or exceeding Node.js implementations—such as completing one million requests in approximately six seconds compared to 47 seconds on typical Express.js setups—while maintaining developer productivity. The mature ecosystem including Entity Framework Core for ORM, Swagger for API documentation, and integrated health-check tooling—supports rapid development cycles, and C\#’s strong static typing enhances code safety by detecting errors at compile time. Although the managed runtime introduces marginally greater memory overhead compared with native frameworks, that cost is offset by improved developer efficiency and maintainability. The decision to implement the API using C\# and ASP .NET Core enabled a robust, maintainable, and scalable backend, and Visual Studio’s seamless Azure integration simplified deployment and lifecycle management through managed pipelines and services.

\subsection{PostgreSQL}
PostgreSQL offers a powerful relational database engine that ensures full ACID compliance through multiversion concurrency control, maintaining data consistency even after unexpected shutdowns. Its support for advanced data types such as JSONB and arrays equips it to handle complex, document-like structures within a single schema, while the PostGIS extension enables comprehensive spatial querying and indexing. High availability is provided through both streaming and logical replication, and tools like Patroni can be employed to automate failover and clustering with minimal runtime intervention. PostgreSQL has a well-documented reputation for reliability in mission-critical settings, with deployments that have sustained 24/7 operation without experiencing data loss or corruption. Its sophisticated query planner and execution engine deliver robust performance across a range of workloads, from high-volume OLTP tasks to computationally heavy analytical queries. Although achieving peak performance may require tuning parameters such as autovacuum thresholds and work memory, and administrators moving into enterprise-scale environments may encounter a steep learning curve, these trade-offs are offset by PostgreSQL’s unwavering data integrity guarantees, scalable architecture, and zero licensing costs. The decision to adopt PostgreSQL provided a dependable and cost-effective foundation capable of managing diverse datasets—including user profiles, session histories, access permissions, and geographic location logs.

\subsection{Microsoft Azure}
Azure delivers both Platform-as-a-Service (PaaS) and Infrastructure-as-a-Service (IaaS) capabilities, supporting App Services, managed PostgreSQL instances, serverless compute via Functions, virtual machines, and storage services. The platform also offers built-in continuous integration and deployment through Azure DevOps or GitHub Actions pipelines, coupled with comprehensive monitoring and diagnostics via Application Insights and Log Analytics. These features enable automatic global scalability, load balancing, patching, and infrastructure maintenance—allowing developers to prioritise application logic over operational tasks—while service-level agreements and regional data centre redundancy provide enterprise-grade availability. Despite its benefits, Azure’s initial configuration demands careful design: defining virtual networks, access policies, and security settings introduces complexity, and the deep integration with proprietary services increases the risk of vendor lock-in, making migration more difficult and costly in the future. Furthermore, while Azure’s pay-as-you-go model enables flexible spending, costs can grow exponentially without constant monitoring and optimisation. Notwithstanding these challenges, Azure provided a secure and scalable hosting environment for the SPLASH web application. Its seamless integration with .NET and Visual Studio simplified deployment processes, and managed services allowed rapid provisioning of production-quality infrastructure through CI/CD pipelines and monitoring tools, ensuring high availability and maintainability of the backend stack.

\subsection{ESP32-S3-WROOM-1}  
The ESP32-S3-WROOM-1 module integrates an Xtensa® dual-core 32-bit LX7 microprocessor that operates at up to 240 \ac{MHz}, delivering sufficient performance to manage concurrent tasks, such as \ac{BLE} communication and \ac{GPS} data parsing. This \ac{SoC} includes an ultra-low-power co-processor that can monitor peripherals while the main cores are powered down, enabling extremely power-efficient operation. Native support for 2.4 \ac{GHz} Wi-Fi (IEEE 802.11 b/g/n) and Bluetooth 5 LE enhances communication flexibility and simplifies printed circuit board design. Peripheral interfaces such as \ac{UART}, \ac{I2C}, \ac{SPI}, PWM, ADC, and a full-speed \ac{USB} 2.0 OTG interface further support rich hardware integration. Security is addressed at the hardware level through secure boot, flash encryption, and integrated cryptographic accelerators for \ac{AES}, \ac{SHA}, \ac{RSA}, and \ac{HMAC}, thereby ensuring firmware integrity and preventing tampering. These hardware features are complemented by the \ac{ESPIDF}, which is based on \ac{FreeRTOS} and includes comprehensive libraries, drivers, and examples that accelerate development. The module’s compact dimensions of 18 × 25.5 × 3.1 mm suit wearable applications. However, this concentration of performance and features increases complexity: configuring low-power modes, peripheral multiplexing, and secure boot procedures requires a thorough understanding of embedded systems, and the rich feature set entails careful power management to maintain acceptable battery life. In the context of this project, the use of the ESP32-S3-WROOM-1 provided a mature and powerful foundation for the wearable device, and its support within \ac{ESPIDF} significantly simplified firmware development and debugging workflows\cite{esp32s3_1}.

\subsection{u-blox NEO-6M}
The u-blox NEO-6M \ac{GPS} module delivers horizontal position accuracy of approximately 2.5 m, ensuring sufficient spatial resolution for wearable localization. It achieves tracking sensitivity down to –161 dBm, enabling reliable satellite acquisition and maintenance even under weak-signal conditions. Time synchronization relies on a \ac{PPS} output with a root-mean-square accuracy of 30 ns, which supports precise timestamping of location data. The module’s navigation update rate reaches up to 5 \ac{Hz}, balancing responsiveness with power consumption for tracking typical human movement. Communication occurs primarily over \ac{UART}, though the NEO-6M also offers \ac{SPI}, \ac{I2C} and \ac{USB} interfaces for integration flexibility. Initial fixes emerge after a cold start of around 27 s, reducing delay for first-time position acquisition. Packaged in a compact 16.0 × 12.2 × 2.4 mm LCC, the module suits the size constraints of a wearable bracelet. All NEO-6 modules are built on GPS chips qualified to the AEC-Q100 automotive standard and comply with RoHS directives, ensuring both reliability in harsh environments and regulatory alignment. Despite these strengths, the maximum 5 \ac{Hz} update rate may limit very high-speed tracking scenarios, and cold-start durations near half a minute can delay initial lock; furthermore, achieving peak performance requires appropriate antenna selection and configuration. The adoption of the NEO-6M provided a cost-effective, rugged, and highly accurate foundation for managing user location profiles, session logs, permissions, and geolocation records within the SPLASH ecosystem\cite{neo6_1,neo6_2}.

\subsection{Bluetooth Low Energy (BLE)}
This section presents the technologies employed for this project. \ac{BLE} defines a wireless standard optimised for intermittent, low-power data exchange at physical-layer bit rates of 125 kbps, 500 kbps, 1 Mbps and 2 Mbps and supports both connection-oriented and broadcast communication modes. The protocol employs adaptive frequency hopping to mitigate interference in the crowded 2.4 \ac{GHz} ISM band by dynamically excluding poor-performing channels based on real-time environmental measurements. \ac{AES}-CCM encryption and message authentication protect data integrity and privacy during transmission, meeting modern security requirements for sensitive information. For ultra-low-power scenarios, \ac{BLE} supports subrated connections that maintain persistent but extremely low-duty-cycle links, minimising energy draw outside of active data-transfer events.  
Because battery conservation is paramount in wearable devices, \ac{BLE} radios consume only a few milliamperes during active packet exchanges and drop to microampere-level sleep currents when idle. The high-speed LE 2M PHY can sustain application-level throughputs up to 1.4 \ac{Mbps}—adequate for periodic location updates and sensor telemetry—while legacy LE 1M and coded PHYs trade peak speed for extended range. \ac{BLE}’s half-duplex radio architecture requires that transmission and reception cannot occur simultaneously, necessitating scheduling strategies for bidirectional data flows. Furthermore, the link-layer state machine’s complexity—driven by advanced quality-of-service, privacy and mesh-networking features—can extend development effort when custom profiles or non-standard workflows are needed. Typical operational range spans 30–50 m in unobstructed environments, satisfying most wearable scenarios, although actual distance varies with antenna design and environmental conditions. \ac{BLE}’s combination of ultra-low active and sleep currents, robust security framework, adaptive interference management and flexible PHY options for both throughput and range makes it ideally suited for the SPLASH bracelet’s requirement to relay periodic location and status information over extended battery lifetimes with minimal additional development overhead\cite{bt_1}.

\subsection{Apache Mynewt NimBLE}
Apache Mynewt NimBLE is a fully open-source \ac{BLE} stack that provides both host and controller implementations under the Apache License 2.0. It implements the complete Bluetooth 5 specification, including L2CAP channels, ATT protocols, GAP for device discovery and advertising, and \ac{GATT} for service and characteristic operations. The stack supports extended and periodic advertising with multiple advertising sets, giving engineers control over payload contents, intervals and physical-layer parameters. Native Bluetooth Mesh support—including PB-GATT and PB-ADV provisioning, relay functionality and GATT Proxy—enables scalable many-to-many topologies that use managed-flood algorithms for efficient low-power relaying. Security is provided end-to-end via bonding procedures, \ac{AES}-CCM link-layer encryption and integrity checks to prevent tampering. Thanks to its modular design and careful resource management, NimBLE operates with a RAM footprint as small as 8–16 KB and flash usage under 100 KB, making it ideal for constrained microcontrollers. Fine-grained power control follows from the ability to tune clock sources and advertising parameters to minimise radio-on time, extending battery life in energy-sensitive applications. Integration with the \ac{ESPIDF} framework on the ESP32 platform delivers a native host port, VHCI interface and \ac{FreeRTOS} compatibility, reducing porting effort and accelerating development cycles. Despite these strengths, achieving optimal energy savings requires careful calibration of clock-settle durations and advertisement intervals, which can lengthen development and testing time, and enabling mesh or address-privacy features without precise configuration can inadvertently increase radio duty cycles and diminish autonomy. By combining a compact footprint, comprehensive GAP/\ac{GATT} APIs, robust security services and exhaustive power-tuning options within an \ac{ESPIDF}–integrated stack, Apache Mynewt NimBLE delivers engineers precise control over \ac{BLE} operations and broad smartphone compatibility, supporting modern IoT and wearable applications\cite{nimble_1,nimble_2}.

%%\section{Conclusion Of The State Of The Art}

\section{Market Analysis and Business Evaluation}
This comprehensive analysis highlights the market potential for SPLASH in improving beach safety while addressing existing gaps in current solutions.

\subsection{Market Overview}
\begin{enumerate}
    \item \textbf{Demand for Beach Safety Solutions:} The rise in tourism necessitates effective safety measures to protect beachgoers, driving demand for innovative solutions.
    \item \textbf{Unique Value Proposition:} SPLASH aims to unify various applications and stakeholder needs into a single platform, enhancing beach safety management.
    \item \textbf{Global Expansion Potential:} Initially targeting Portuguese beaches, SPLASH's scalability allows for future international market entry.
    \item \textbf{Monetization Strategies:} The business model includes premium accounts for local businesses and a licensing model for lifeguard organizations, ensuring sustainable revenue streams.
\end{enumerate}

\subsection{Competitors Analysis}
The competitive landscape includes established services and emerging technologies.

\subsubsection{Current Technologies}
\begin{itemize}
    \item \ac{apa}: Conducts water quality assessments and operates the \textit{Beachcam} web app for live feeds of Portuguese beaches.
    \item \ac{amn}: Runs lifesaving stations and projects like \textit{Project Seawatch} focused on surveillance in unmonitored areas.
    \item Info Praia App: Provides updates on water quality but lacks availability on major platforms.
\end{itemize}

\subsubsection{Emerging Technologies}
\begin{itemize}
    \item AI-Based Riptide Detection Tools
    \item Smart Beach Initiatives
    \item Drone Surveillance Systems
\end{itemize}

SPLASH distinguishes itself by integrating these technologies into a cohesive platform focused on proactive rather than reactive safety measures.

\section{Business Model Evaluation}
SPLASH employs a multi-faceted business model addressing the needs of both beachgoers and lifeguard organizations:

\begin{itemize}
    \item \textbf{Free Web Application for Users:} This approach encourages widespread adoption while collecting valuable user data.
    \item \textbf{Premium Listings for Businesses:} Local businesses can enhance visibility within the app, benefiting from increased traffic while supporting SPLASH financially.
    \item \textbf{Licensing for Lifeguards:} Affordable licensing options will be offered to lifeguard organizations once established, providing access to advanced features tailored to their needs.
\end{itemize}

This model fosters user engagement and creates diverse revenue streams that support long-term growth.

\section{Marketing Strategy Assessment}
To penetrate the market effectively, SPLASH can implement a robust marketing strategy:

\begin{itemize}
    \item \textbf{Partnerships with Authorities:} Collaborating with local governments and tourism boards will promote SPLASH as an essential tool for beach safety, and give SPLASH the needed credibility in the eyes of the general public.
    \item \textbf{Social Media Engagement:} Creating content that emphasizes beach safety and showcases SPLASH's unique features will build brand awareness.
    \item \textbf{Industry Participation:} Attending conferences focused on coastal management will facilitate networking opportunities and demonstrate SPLASH's capabilities.
\end{itemize}

\texttt{\textit{\textbf{Note:} Key Performance Indicators (KPIs) will be utilized to measure user adoption rates, engagement metrics, and feedback.}}

\section{SWOT Analysis}
The following SWOT analysis provides insight into SPLASH's strategic position:

\begin{itemize}
    \item \textbf{Strengths:}
        \begin{itemize}
            \item Innovative technology integration.
            \item Comprehensive approach to beach safety.
            \item Potential positive impact on public health.
        \end{itemize}

    \item \textbf{Weaknesses:}
        \begin{itemize}
            \item Initial reliance on user adoption; user-centered development will mitigate resistance by tailoring solutions to user needs.
        \end{itemize}

    \item \textbf{Opportunities:}
        \begin{itemize}
            \item Expanding global tourism market.
            \item Increasing focus on preventive safety measures.
            \item Potential partnerships with stakeholders.
        \end{itemize}

    \item \textbf{Threats:}
        \begin{itemize}
            \item Regulatory challenges across jurisdictions.
            \item Competition from established technologies.
            \item Economic fluctuations impacting tourism spending.
        \end{itemize}    
\end{itemize}