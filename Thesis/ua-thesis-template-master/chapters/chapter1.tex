\chapter{Comprehensive Project Overview}
\label{chapter:overview}

\section{Introduction}
\label{section:intro}
Beaches are a universal attraction, and it is widely recognized that an increasing number of people travel specifically to experience them. Portugal, renowned for its exceptional weather and ideal coastal conditions, has become a prime destination for this growing wave of tourism. This trend has significantly increased the workload for lifeguards, who are now tasked with managing larger crowds. As a result, the pressure on lifeguards has escalated, potentially compromising their ability to respond effectively to emergencies and increasing the risks faced by beachgoers.
\\
The \ac{splash} project is a proof of concept that addresses these challenges as Portugal’s beaches attract increasing numbers of tourists each year, aiming to enhance the management of Portuguese beaches through a preventive technological solution system that mitigates risks in real time.
\\
\ac{splash} contributes to the modernization of coastal safety practices, safeguarding human lives and reinforcing Portugal's position as a global benchmark in beach safety and sustainability, ensuring its beaches remain safe, efficient, and welcoming for future generations of visitors.
\\
In this chapter we will provide a overview of the project, encompassing its \textbf{context}, \textbf{challenges}, \textbf{objectives}, \textbf{boundaries}, and \textbf{expected outcomes}. By examining these crucial aspects, we aim to offer a clear and concise understanding of the project scope and significance.

\section{Context}
\label{section:context}

This project, focused on hazard prevention and the enhancement of lifeguard operations, was developed at the \ac{ua} as part of the curricular unit \ac{peci}, which serves as the culminating initiative for the bachelor's degree in \ac{eci}. The project is being carried out by a team of six students who collaboratively proposed the theme, demonstrating a proactive approach to addressing real-world challenges in beach safety. 
\\
The initiative highlights the team's commitment to leveraging their academic knowledge and technical skills to design an innovative solution that integrates preventive measures with technological systems, ensuring both practicality and impact. Thus, while the safety of beachgoers is essential, lifeguards often face a shortage of human resources, particularly during peak seasons. To address these challenges, technological and intelligent systems can play a crucial role in significantly improving beach safety.


\section{Challenges}
\label{section:challenges}
The project aims to tackle several key challenges faced by lifeguards:

    \begin{itemize}
        \item Monitoring and controlling large beach areas.
        \item Team communication, schedule and resource management.
        \item Detection and response times to dangerous or emergency situations.
        \item Sudden climate changes that can affect beach safety.
    \end{itemize}

\section{Objectives}
\label{section:objectives}
The main objective of \ac{splash} is to provide a range of features that support the different dimensions entailed in a lifeguards work
such as technological features and preventive measures, assisting lifeguards to perform their tasks and duties. This primary goal is supported by several specific objectives: \\ 

Design and develop a technological solution to:
\begin{itemize}
    \item Improve surveillance and control of large beach areas.
    \item Reduce detection and response time to emergency situations. 
    \item Enhance lifeguard team coordination.
    \item Implement an efficient resource management system for lifeguards.
    \item Provide curated safety information to the public.
\end{itemize}
      
\section{Boundaries}
\label{section:boundaries}

\textbf{\ac{splash} is designed to provide the following capabilities: }
\\
    \begin{itemize}
        \item Available on beaches, focusing on coastal safety.
        \item Support lifeguards in their duties.
        \item Hazard prevention and information transmission.
     \item Accessible to various stakeholders. 
\end{itemize}

\vspace{1em}
\textbf{The system is designed to have the following key features:}
\\
    \begin{itemize}
        \item Interactive map
        \item Danger report and signalling.
        \item Real-time information.
        \item Beach history and metrics information.
        \item Lifeguard communication system.
        \item Beach amenities information.     
    \end{itemize}
    
\vspace{1em}
\textbf{The target audience of the system is made up of three main user groups:}
\\
    \begin{itemize}
        \item \textbf{Lifeguards:} Supervisors and regular team members.
        \item \textbf{Beachgoers:} People who visit the beach for leisure, swimming, sunbathing, etc.
        \item \textbf{Beach business owners:} Like surf schools, amenity providers, food vendors, equipment rental shops, etc.
    \end{itemize}
    
\section{Expected Outcomes}
\label{section:expected_outcomes}

    \begin{itemize}
        \item \textbf{For lifeguards:} A tool for managing resources, controlling beach areas, and facilitating team communication.
        \item \textbf{For beachgoers:} A resource for safety awareness, navigation assistance, and easy access to information about beach facilities.
        \item \textbf{For beach business owners:} A platform that increases visibility to potential customers, provides location details, and offers targeted marketing opportunities.
    \end{itemize}