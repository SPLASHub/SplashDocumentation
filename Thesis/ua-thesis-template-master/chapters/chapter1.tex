\chapter{Project Overview}
\label{chapter:overview}

\section{Introduction}
\label{section:intro}

Beaches are universally popular destinations, and the number of tourists seeking coastal experiences continues to rise. Portugal, known for its favourable climate and extensive coastline, has become a prime location for this growing trend. As a result, lifeguards are under increasing pressure to manage larger crowds, which can compromise their ability to respond effectively to emergencies and heighten the risks faced by beachgoers.




The \ac{splash} project is a proof of concept developed in response to these challenges. It proposes a preventive and technological approach to support beach safety and management in real time. By leveraging digital tools, SPLASH aims to assist lifeguards in performing their duties more effectively, while also promoting safer behaviours among the public.

This initiative contributes to the modernization of coastal safety practices, helping to safeguard human lives and reinforcing Portugal’s reputation as a global benchmark in beach sustainability and safety.

This chapter outlines the project's \textbf{context}, \textbf{challenges}, \textbf{objectives}, \textbf{boundaries}, and \textbf{expected outcomes}, providing a comprehensive overview of its scope and relevance.

\section{Context}
\label{section:context}

This project, focused on hazard signalling, team coordination, and preventive safety, was developed at the \ac{ua} as part of the \ac{peci} course, marking the final stage of the bachelor's degree in \ac{eci}. It was carried out by a team of six students who independently selected the topic, demonstrating both initiative and a strong interest in addressing real-world issues related to beach safety.

The work reflects the application of knowledge acquired throughout the degree in a practical context. With beach safety becoming an increasing concern, particularly during summer when lifeguard staffing is often limited, this project explores how smart systems and digital tools can enhance lifeguard operations and contribute to safer beach environments.

\section{Challenges}
\label{section:challenges}

The project aims to address several key challenges faced by lifeguards:

\begin{itemize}
    \item Inefficient communication, scheduling, and resource allocation.
    \item Difficulty monitoring and managing large beach areas.
    \item Delayed detection and signalling of hazards and dangerous conditions.
    \item Increasing climate variability impacting beach safety.
\end{itemize}

\section{Objectives}
\label{section:objectives}

The primary objective of \ac{splash} is to develop a technological solution that supports the various dimensions of a lifeguard's responsibilities, through preventive strategies and digital tools. The specific objectives include:

\begin{itemize}
    \item \textbf{Support human resource management:} Provide tools for supervisors to assign lifeguards to posts, manage weekly schedules, and monitor real-time duty status, thus improving operational efficiency.
    
    \item \textbf{Enable hazard reporting and tracking:} Allow lifeguards to report hazards using an interactive map, with logs per beach to support data-driven preventive action.
    
    \item \textbf{Provide weather information:} Integrate real-time weather data and forecasts to support safety decisions under changing conditions.
    
    \item \textbf{Inform the public:} Offer a simplified version of the system to beach-goers, showing up-to-date conditions and hazard alerts, promoting awareness, and reducing unnecessary interactions with lifeguards.

    \item \textbf{Enable device integration within a single ecosystem:} Facilitate seamless onboarding and management of external IoT devices in the \ac{splash} ecosystem, allowing real-time data streams such as location or environmental detection to enrich situational awareness and improve prevention and response capabilities benefiting lifeguards and bathers.

\end{itemize}

\section{Boundaries}
\label{section:boundaries}

As a proof of concept, the \ac{splash} system focuses on supporting lifeguards, improving communication of beach hazards, and informing beach-goers. The system is designed to:

\begin{itemize}
    \item Be deployed in beach settings with a focus on coastal safety.
    \item Assist lifeguards in their operational duties.
    \item Facilitate hazard detection, reporting, and information sharing.
    \item Serve multiple stakeholders, including supervisors, tourists, and beach-related businesses.
\end{itemize}

\noindent The core functionalities of the system include:
\begin{itemize}
    \item Interactive map with real-time data.
    \item Hazard reporting and alert mechanisms.
    \item Historical beach metrics and incident logs.
    \item Information on available beach amenities.
\end{itemize}

\noindent The system is intended for the following user groups:
\begin{itemize}
    \item \textbf{Lifeguards:} Both supervisors and general staff.
    \item \textbf{Beach-goers:} Individuals visiting for leisure.
    \item \textbf{Beach business owners:} Surf schools, food stalls, rentals, and related services.
\end{itemize}

\section{Expected Outcomes}
\label{section:expected_outcomes}
Based on the objectives and system boundaries defined above, this subsection describes the tangible benefits that the \ac{splash} proof of concept will provide to its primary stakeholders.
\begin{itemize}
    \item \textbf{For lifeguards:} A digital platform to streamline human resource management, allowing assignment to posts, scheduling organization, and real-time tracking of on-duty staff. The system will also support hazard reporting, store incident histories, and integrate weather forecasts to enhance situational awareness.
    
    \item \textbf{For beach-goers:} A simplified public interface providing access to current weather conditions and hazard alerts for each beach. This promotes safer decision making, encourages responsible behaviour, and raises awareness of potential risks.
\end{itemize}
