\chapter{Future Work and Contributions}
\label{chapter:Future}

Although the SPLASH system achieved its initial objectives and demonstrated high usability, there are several directions in which the platform can be further developed and improved. These enhancements are essential to increase the system's robustness, scalability, and overall value to its various stakeholders.

\section{Proposed Future Enhancements}

\subsection{Real-Time Messaging System}
One of the most impactful additions to the SPLASH platform would be the implementation of a dedicated messaging system, enabling real-time communication between lifeguards, supervisors, and potentially beachgoers. This feature would streamline coordination, facilitate the rapid exchange of information during emergencies, and reduce the reliance on external communication tools (e.g., radios or mobile phones). Key capabilities could include:
\begin{itemize}
    \item Private and group chat channels for lifeguard teams
    \item Priority tagging for critical messages (e.g., emergency alerts)
    \item Integrated notification system linked to the interactive map
\end{itemize}

\subsection{Dedicated System for Field Lifeguards}
The current proof of concept already supports supervisory roles and coordination tasks; however, a dedicated and optimized interface for lifeguards operating directly on the beach would further enhance usability in the field. This system should prioritize minimal interaction time, fast access to incident reports, and simplified tools for:
\begin{itemize}
    \item Confirming or dismissing hazard alerts
    \item Accessing shift schedules and team assignments
    \item Logging incidents and responding to GPS-based alerts
\end{itemize}
Additionally, such a system could be implemented as a progressive web app or mobile application to ensure compatibility with tablets or smartwatches.

\subsection{Expanded Functionality for Beachgoers}
While the current system offers beachgoers access to hazard information and basic alerts, additional features could increase engagement, safety, and satisfaction. Potential enhancements include:
\begin{itemize}
    \item Personalized risk notifications based on location and behavior
    \item Expanded beach service directory with filters and reviews
    \item Gamified safety challenges or educational prompts
    \item Real-time feedback channels for reporting issues or suggestions
\end{itemize}
These improvements would support a more participatory safety ecosystem and encourage responsible behavior among beach users.

\section{Broader Contributions}
Beyond its technical achievements, the SPLASH project offers broader contributions to coastal safety and digital public services:

\begin{itemize}
    \item \textbf{Scalable Architecture}: The system’s modular design supports future expansion to new beaches, municipalities, or even international contexts.
    \item \textbf{User-Centered Methodology}: The project reinforces the value of participatory design in public safety systems, particularly in high-stakes environments such as coastal surveillance.
    \item \textbf{Educational Impact}: The project has increased awareness around coastal hazards and the operational challenges faced by lifeguards, highlighting opportunities for innovation in public safety infrastructure.
\end{itemize}

\section{Conclusion}
The SPLASH platform provides a strong foundation for improving beach safety through technology. However, its long-term success depends on continuous improvement, stakeholder collaboration, and the adoption of advanced features that address the evolving needs of lifeguards and beachgoers alike. The future work outlined in this chapter represents the next logical step in transforming SPLASH from a proof of concept into a comprehensive operational solution.
