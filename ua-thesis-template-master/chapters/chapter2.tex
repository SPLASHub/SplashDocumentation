\chapter{State of the Art and Market Analysis}
\label{chapter:sota}

This chapter merges the current state of technology in beach safety with a comprehensive market analysis, focusing on the SPLASH project and its potential impact.

\section{Related Projects Implemented in Portugal}
\subsection{MEO Beachcam}
MEO Beachcam is a digital platform that provides real-time information about the beaches of mainland Portugal.
\\
\indent The platform provides around the clock video transmission of the beaches through a network of cameras. In addition to the live video feeds, MEO Beachcam provides extensive meteorological data, including: weather conditions, temperature, wind strength and direction, sea state and UV index. Besides being accessible on the platform, the aforementioned data is also available to security forces and rescue services for monitoring and emergency response. In addition to the oceanographic buoys and weather stations for data collection, the platform also employs the use of Artificial Intelligence for analyzing wave patterns and better predict weather conditions. 
\\
\indent The project is committed to expand and enhance its coastal monitoring infrastructure. Future plans for development include the expansion of the existing camera network and the deployment of drone technology to capture aerial imagery, all of this while seeking to increase the number of monitored beaches throughout mainland Portugal.

\subsection{Info Praia}
Info Praia is a mobile app developed by the \ac{apa}  with the purpose of provide actualized information during the bathing season on many of the beaches of Portugal.
\\
\indent It provides information on the beach occupation, and occupation prediction based on data of registered homologues days. The data used for the daily weather forecast is provided by the Instituto Portugues do Mar e da Atmosfera, I.P (IPMA, I.P). The data about the tides  is provided by the Instituto Hidrográfico. The water quality is monitored by the APA at least four times for each bathing season to verify the bathing suitability and display its suitability in the app. Furthermore, is displayed by the app real-time video transmitted by the MEO Beachcam platform analyzed previously.
\subsection{Praia 5G}
Praia 5G is a joint initiative between NOS and Almada city council to launch the first 5G beach in Portugal. This project has the objective of demonstrating how the 5G network technology, artificial intelligence and video analysis can transform the security and management of beaches.
\\
\indent The beach was equipped with a 5G network to support all the functionalities implemented. One of the functionalities is the automated monitorization of persons, objects and vessels. This functionality includes the detection of risk situations, automated alerts for the responsible entities and it uses artificial vision cameras in vigilance towers that process in real time the images through machine learning algorithms allowing immediate response in emergency situations. Another functionality is the monitorization and management of the beach occupation which includes automatic counting the number of visitors and their respective entry and exit times and the capacity to estimate the concentration of people per area of the beach. Lastly, the management of residues is elaborated by the readings of sensors that measure the garbage bins capacity.
\\
\indent This project has the future perspective of implementing functionalities such as drowning detection, children and lost objects localization through artificial intelligence. Also, it is intended to use drones for real time data of weather conditions and to increase the sustainability expand the system to detect garbage in the sea with the possibility of the use of cleaning robots.

\iffalse
\section{Others Related Projects Implemented Internationally}
\section{Technologies Of Interest}
\section{Conclusion Of The State Of The Art}
\fi

\section{Market Analysis and Business Evaluation}
This comprehensive analysis highlights the market potential for SPLASH in improving beach safety while addressing existing gaps in current solutions.

\subsection{Market Overview}
\begin{enumerate}
    \item \textbf{Demand for Beach Safety Solutions:} The rise in tourism necessitates effective safety measures to protect beachgoers, driving demand for innovative solutions.
    \item \textbf{Unique Value Proposition:} SPLASH aims to unify various applications and stakeholder needs into a single platform, enhancing beach safety management.
    \item \textbf{Global Expansion Potential:} Initially targeting Portuguese beaches, SPLASH's scalability allows for future international market entry.
    \item \textbf{Monetization Strategies:} The business model includes premium accounts for local businesses and a licensing model for lifeguard organizations, ensuring sustainable revenue streams.
\end{enumerate}

\subsection{Competitors Analysis}
The competitive landscape includes established services and emerging technologies.

\subsubsection{Current Technologies}
\begin{itemize}
    \item \ac{apa}: Conducts water quality assessments and operates the \textit{Beachcam} web app for live feeds of Portuguese beaches.
    \item \ac{amn}: Runs lifesaving stations and projects like \textit{Project Seawatch} focused on surveillance in unmonitored areas.
    \item Info Praia App: Provides updates on water quality but lacks availability on major platforms.
\end{itemize}

\subsubsection{Emerging Technologies}
\begin{itemize}
    \item AI-Based Riptide Detection Tools
    \item Smart Beach Initiatives
    \item Drone Surveillance Systems
\end{itemize}

SPLASH distinguishes itself by integrating these technologies into a cohesive platform focused on proactive rather than reactive safety measures.

\section{Business Model Evaluation}
SPLASH employs a multi-faceted business model addressing the needs of both beachgoers and lifeguard organizations:

\begin{itemize}
    \item \textbf{Free Web Application for Users:} This approach encourages widespread adoption while collecting valuable user data.
    \item \textbf{Premium Listings for Businesses:} Local businesses can enhance visibility within the app, benefiting from increased traffic while supporting SPLASH financially.
    \item \textbf{Licensing for Lifeguards:} Affordable licensing options will be offered to lifeguard organizations once established, providing access to advanced features tailored to their needs.
\end{itemize}

This model fosters user engagement and creates diverse revenue streams that support long-term growth.

\section{Marketing Strategy Assessment}
To penetrate the market effectively, SPLASH can implement a robust marketing strategy:

\begin{itemize}
    \item \textbf{Partnerships with Authorities:} Collaborating with local governments and tourism boards will promote SPLASH as an essential tool for beach safety, and give SPLASH the needed credibility in the eyes of the general public.
    \item \textbf{Social Media Engagement:} Creating content that emphasizes beach safety and showcases SPLASH's unique features will build brand awareness.
    \item \textbf{Industry Participation:} Attending conferences focused on coastal management will facilitate networking opportunities and demonstrate SPLASH's capabilities.
\end{itemize}

\texttt{\textit{\textbf{Note:} Key Performance Indicators (KPIs) will be utilized to measure user adoption rates, engagement metrics, and feedback.}}

\section{SWOT Analysis}
The following SWOT analysis provides insight into SPLASH's strategic position:

\begin{itemize}
    \item \textbf{Strengths:}
        \begin{itemize}
            \item Innovative technology integration.
            \item Comprehensive approach to beach safety.
            \item Potential positive impact on public health.
        \end{itemize}

    \item \textbf{Weaknesses:}
        \begin{itemize}
            \item Initial reliance on user adoption; user-centered development will mitigate resistance by tailoring solutions to user needs.
        \end{itemize}

    \item \textbf{Opportunities:}
        \begin{itemize}
            \item Expanding global tourism market.
            \item Increasing focus on preventive safety measures.
            \item Potential partnerships with stakeholders.
        \end{itemize}

    \item \textbf{Threats:}
        \begin{itemize}
            \item Regulatory challenges across jurisdictions.
            \item Competition from established technologies.
            \item Economic fluctuations impacting tourism spending.
        \end{itemize}    
\end{itemize}