\chapter{Introdução}%
\label{chapter:introduction}

\begin{introduction}
\ac{splash} é um projecto concebido para melhorar a segurança e gestão das praias Portuguesas, vitais para o turismo e a economia nacional. Devido a afluência anual de milhares de turistas, garantir a segurança dos banhistas é fundamental. Esta iniciativa aborda os desafios enfrentados por nadadores-salvadores, especialmente durante a época alta, recorrendo a uma solução inteligente com integração de tecnologias avançadas. \\
Tal como \textit{Herman Melville} observou em \textit{"Moby Dick": "Considere a subtileza do mar; como as suas criaturas mais temidas deslizam sob a água, impercetíveis na maior parte do tempo, e traiçoeiramente escondidas sob os mais adoráveis tons de azul"}, o \ac{splash} incorpora esta perspetiva de prevenção e vigilância, reconhecendo que os perigos nas praias nem sempre são percetíveis à superfície. Assim, tal como o mar esconde suas ameaças, o projeto visa utilizar uma solução inteligente para detetar e prevenir riscos que podem passar despercebidos, garantindo uma experiência segura para todos os banhistas.
\end{introduction}



\section{Acrónimos}

Primeira e seguintes referências: \ac{h2o}, \ac{h2o}

Plural, acrónimo expandido e curto: \acp{h2o}, \acl{h2o}, \acs{h2o}

Com citação\footnote{Necessária entrada na bibliografia}: \ac{adsl}, \ac{adsl}


\section{Fontes}

\begin{itemize}
\item{\tiny Tiny}
\item{\scriptsize Scriptsize}
\item{\footnotesize Footnotes}
\item{\small Small}
\item{\normalsize Normal}
\item{\large large}
\item{\Large Large}
\item{\LARGE LARGE}
\item{\huge huge}
\item{\Huge Huge}
\end{itemize}

\section{Unidades}

Utilizando o pacote \verb|siunitx| é possível utilizar unidades do Sistema Internacional. Exemplo: a aceleração da gravidade é de \SI{9.8}{\metre\per\second\squared} e um ficheiro ocupa \SI{1}{\mebi\byte}. 

\section{Code Blocks}
%\lipsum[5]
Uma listagem pode ser apresentada com o ambiente \texttt{listing}, que é um float (objeto flutuante, tal como uma figura ou uma tabela).

A listagem em Código~\ref{lbl:snippet-test} mostra um exemplo em C.

\begin{listing}[h]
\begin{minted}{c}

#include <stdio.h>
#define N 10
/* Block
 * comment */
 
int main()
{
    int i;
 
    // Line comment.
    puts("Hello world!");
 
    for (i = 0; i < N; i++)
    {
        puts("LaTeX is also great for programmers!");
    }
 
    return 0;
}
\end{minted}
\caption{This caption appears below the code.}
\label{lbl:snippet-test}
\end{listing}

%\lipsum[5]

\section{Citações}

Algumas formas distintas de citar:

\begin{itemize}
    \item \textbf{Apenas referência}:~\cite{rfc44}
    \item \textbf{Apenas data}:~\citedate{rfc44}
    \item \textbf{Apenas ano}:~\citeyear{rfc44}
    \item \textbf{Apenas autor}:~\citeauthor{rfc44}
    \item \textbf{Apenas editor}:\citelist{rfc44}{organization}
    \item \textbf{Autor e referência}:\citet{rfc44}
\end{itemize}
