\chapter{Introdução}%
\label{chapter:introduction}

\begin{introduction}
O projecto \ac{splash} é uma prova de conceito inovadora concebida para melhorar a segurança e gestão das praias Portuguesas, elementos vitais para o turismo e a economia nacional. Devido à afluência anual de milhares de turistas, garantir a segurança dos banhistas torna-se uma prioridade fundamental. Esta iniciativa aborda os desafios enfrentados por nadadores-salvadores, especialmente durante a época alta, recorrendo a uma solução inteligente que integra tecnologias avançadas. \\
Tal como Luís de Camões escreveu nos Lusíadas, Canto V, estrofe 16: \\ 
\textit{\textbf{"\begin{center}
    "Contar-te longamente as perigosas, \\
    Cousas do mar, que os homens não entendem, \\
    Súbitas trovoadas temerosas, \\
    Relâmpados que o ar em fogo acendem, \\
    Negros chuveiros, noites tenebrosas, \\
    Bramidos de trovões, que o mundo fendem, \\
    Não menos é trabalho que grande erro, \\
    Ainda que tivesse a voz de ferro" \\
\end{center}}}, 

O \ac{splash} incorpora esta perspetiva de enfrentar os desafios, reconhecendo que os perigos nas praias podem ser de variados e por vezes imprevisíveis. O objetivo é criar um sistema inteligente preventivo e robusto o suficiente por forma a mitigar os riscos da prática balnear. \\
Esta abordagem inovadora não só honra a tradição marítima portuguesa celebrada por Camões, mas também a projeta para o futuro, garantindo que as praias de Portugal continuem a ser um destino seguro e acolhedor para todos os visitantes. \\
\end{introduction}



\section{Acrónimos}

Primeira e seguintes referências: \ac{h2o}, \ac{h2o}

Plural, acrónimo expandido e curto: \acp{h2o}, \acl{h2o}, \acs{h2o}

Com citação\footnote{Necessária entrada na bibliografia}: \ac{adsl}, \ac{adsl}


\section{Fontes}

\begin{itemize}
\item{\tiny Tiny}
\item{\scriptsize Scriptsize}
\item{\footnotesize Footnotes}
\item{\small Small}
\item{\normalsize Normal}
\item{\large large}
\item{\Large Large}
\item{\LARGE LARGE}
\item{\huge huge}
\item{\Huge Huge}
\end{itemize}

\section{Unidades}

Utilizando o pacote \verb|siunitx| é possível utilizar unidades do Sistema Internacional. Exemplo: a aceleração da gravidade é de \SI{9.8}{\metre\per\second\squared} e um ficheiro ocupa \SI{1}{\mebi\byte}. 

\section{Code Blocks}
%\lipsum[5]
Uma listagem pode ser apresentada com o ambiente \texttt{listing}, que é um float (objeto flutuante, tal como uma figura ou uma tabela).

A listagem em Código~\ref{lbl:snippet-test} mostra um exemplo em C.

\begin{listing}[h]
\begin{minted}{c}

#include <stdio.h>
#define N 10
/* Block
 * comment */
 
int main()
{
    int i;
 
    // Line comment.
    puts("Hello world!");
 
    for (i = 0; i < N; i++)
    {
        puts("LaTeX is also great for programmers!");
    }
 
    return 0;
}
\end{minted}
\caption{This caption appears below the code.}
\label{lbl:snippet-test}
\end{listing}

%\lipsum[5]

\section{Citações}

Algumas formas distintas de citar:

\begin{itemize}
    \item \textbf{Apenas referência}:~\cite{rfc44}
    \item \textbf{Apenas data}:~\citedate{rfc44}
    \item \textbf{Apenas ano}:~\citeyear{rfc44}
    \item \textbf{Apenas autor}:~\citeauthor{rfc44}
    \item \textbf{Apenas editor}:\citelist{rfc44}{organization}
    \item \textbf{Autor e referência}:\citet{rfc44}
\end{itemize}
